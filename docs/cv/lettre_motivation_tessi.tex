\documentclass[11pt, a4paper]{article}

\usepackage[T1]{fontenc}
\usepackage[utf8]{inputenc}
\usepackage[top=2.5cm, bottom=2.5cm, left=2.5cm, right=2.5cm]{geometry}
\usepackage{charter}
\usepackage{hyperref}
\usepackage{xcolor}
\hypersetup{colorlinks=true, urlcolor=black}

\pagestyle{empty}
\setlength{\parindent}{0pt}
\setlength{\parskip}{0.6em}

\begin{document}

\begin{flushright}
\textbf{Phil Georges BALOG}\\
phil.balog13@gmail.com\\
07 46 49 67 49\\
\vspace{0.3cm}
Bordeaux, le \today
\end{flushright}

\vspace{0.5cm}

\textbf{BIP TESSI -- Site de Bordeaux Le Haillan}\\
Objet : Candidature au poste de Chargé de Traitement Sécurité Financière (CDD)

\vspace{0.8cm}

Madame, Monsieur,

La lutte contre le blanchiment n'est pas un sujet que j'ai découvert dans une fiche de poste. C'est un sujet qui me passionne depuis que j'ai compris, en construisant de mes propres mains une plateforme monétique complète, à quel point chaque transaction raconte une histoire --- et qu'il suffit parfois d'une incohérence pour révéler un schéma frauduleux.

Dans le cadre d'un projet personnel ambitieux, j'ai conçu un système bancaire simulant le cycle transactionnel de bout en bout : émission de cartes, autorisation, scoring fraude AML, vérification KYC/KYB, écriture comptable et compensation. J'y ai implémenté un pipeline d'analyse en 10 étapes où chaque opération est passée au crible : origine des fonds, cohérence avec le profil client, détection de montants inhabituels, alertes automatiques. C'est exactement le réflexe analytique que demande votre poste : étudier l'origine, la destination et la justification économique d'une transaction, puis décider si le dossier tient la route.

Cette capacité d'analyse, je l'ai aussi forgée en entreprise. Chez Peugeot, j'ai structuré et fiabilisé des données pour produire des recommandations exploitables. Au Ministère des Finances du Cameroun, j'ai audité des configurations de sécurité et mis en place des règles de détection d'anomalies. Chez ENEO, j'ai passé au crible des comptes Active Directory --- vérifier des accès, identifier des écarts, formuler des recommandations et rendre compte de mon travail, c'est ce que je fais depuis mes premiers stages.

Ce qui m'attire chez Tessi, c'est l'exigence. Travailler pour des grands comptes bancaires et assurantiels sur des sujets de conformité réelle, avec un vrai accompagnement à la montée en compétences, c'est l'environnement dans lequel je veux évoluer. Je suis rigoureux, à l'aise avec les outils informatiques, et surtout, je suis quelqu'un qui creuse un dossier jusqu'à ce que chaque pièce s'emboîte --- ou jusqu'à ce que je puisse expliquer pourquoi elle ne s'emboîte pas.

Je serais ravi d'échanger avec vous sur la manière dont mon profil peut contribuer à votre équipe de sécurité financière.

\vspace{0.5cm}

Cordialement,

\vspace{0.8cm}

\textbf{Phil Georges BALOG}

\end{document}
